\documentclass[a4paper,12pt]{article}
\usepackage{geometry}
\usepackage{fancyhdr}
\usepackage{mathtools}


\usepackage{graphicx}
\usepackage{amsmath}
\geometry{
  a4paper,
  total={170mm,257mm},
  left=20mm,
  top=20mm,
}

\usepackage{setspace}

\renewcommand{\headrulewidth}{0pt}
\renewcommand{\footrulewidth}{0pt}
\setlength{\headheight}{30pt}
\fancyhf{}
\rhead{Suban Shrestha\\ 076BCT082} % Name \\ Rollnumber
\lhead{Theory Of Computation} % Subject
\rfoot{Page \thepage}
\pagestyle{fancy}
\onehalfspacing
\begin{document}


\begin{enumerate}
\item
    Determine whether each of the follwing is true or false.
    \begin{enumerate}
        \item
            $\emptyset \subseteq \emptyset$ \\
            True. Everyset is a subset of itself
        \item
            $\emptyset \in \emptyset$ \\
            False. As an empty set cannot contain another set, in this case $\emptyset$.
        \item
            $\emptyset \in \{\emptyset\}$ \\
            True. As the set has the element $\emptyset$
        \item
            $\emptyset \subseteq \{\emptyset\}$ \\
            True. Emptyset is the subset of every set.
        \item
            $\{a, b\} \in \{a, b, c, \{a, b\}\}$ \\
            True. The set contains both elements a and b.
        \item
            $\{a, b\} \subseteq \{a, b, \{a, b\}\}$ \\
            True. As the set $\{a, b\}$ has elements a, b from $\{a, b, \{a, b\}\}$ making it a subset.
        \item
            $\{a, b\} \subseteq 2^{\{a, b, \{a, b\}\}}$ \\ 
            False. Power set is a set of sets and it doesn't have elemtns a and b.
        \item 
            $\{\{a, b \}\} \in 2^{\{a, b, \{a, b \}\}}$ \\
            True. The power set does have a element that is a set of set containing elemetns a and b.
        \item
            $\{a, b, \{a, b \}\} - \{a, b\} = \{a, b\}$ \\
            False. After the operation on the left hand side we are left with $\{\{a, b\}\}$ which is not equal to $\{a, b\}$
    \end{enumerate}
\item
    What are these sets? Write them using brackets, commas and numerals only.
    \begin{enumerate}
        \item
            $(\{1, 3, 5\} \cup \{3, 1\}) \cap \{3, 5, 7 \} = \{3, 5\}$ 
        \item 
            $\cup \{\{3\}, \{3, 5\}, \cap \{\{5, 7\}, \{7, 9\}\}\} = \{3, 5, 7\}$
        \item
            $(\{1, 2, 5\} - \{5, 7, 9\}) \cup (\{5, 7, 9\} - \{1, 2, 5\}) = \{1, 2, 7, 9\}$
        \item
            $2^{\{7, 8, 9\}} - 2^{\{7, 9\}} = \{\{8\}, \{7, 8\}, \{8, 9\}, \{7, 8, 9\}\}$
        \item
            $2^{\emptyset} = \{\emptyset\}$
            
    \end{enumerate}

\pagebreak
\item 
    Prove each of the following
    \begin{enumerate}
        \item
            $A \cup (B \cap C) = (A \cup B) \cap (A \cup C)$ \\ 
            Let, \\ $L = A \cup (B \cap C)$ and $R = (A \cup B) \cap (A \cup C)$ \\
            We are to show that $L = R$. We do this by showing (i) $L \subseteq R$ and (ii) $R \subseteq L$
            \begin{enumerate}
                \item
                    Let x be an element of L. Then either, $x \in A$ or $x \in (B \cap C)$\\
                    if $x \in A$, then this emplies, $x \in (A \cup B)$ and $x \in (A \cup C)$. And therefore, $x \in ((A \cup B) \cap (A \cup C))$.
                    Similarly, if $x \in (B \cap C)$, this emplies that $x \in B$ and $x \in C$.Thus, $x \in (A \cup B)$ and $x \in (A \cup C)$. And hence,  $x \in ((A \cup B) \cap (A \cup C))$.

                    Therefore $L \subseteq R$.
                \item
                    Let $x \in R$, then x in an element of both $A \cup B$ or $A \cup C$. \\
                    If, $x \in A$, then $x \in L$. Similarly, if $x \not\in A$, then x must be in both B and C. i.e $x \in B$ and $x \in C$. Then $x in (B \cap C)$ and thus $x \in L$. \\
                    Therefore $R \subseteq L$. \\
                    Hence we established $L = R$ \\
            \end{enumerate}
        \item
            $A \cap (B \cup C) = (A \cap B) \cup (A \cap C)$ \\
            Let,\\ $L = A \cap (B \cup C)$ and $R = (A \cap B) \cup (A \cap C)$ \\ 
            We are to show that $L = R$. We do this by showing (i) $L \subseteq R$ and (ii) $R \subseteq L$
            \begin{enumerate}
                \item
                    Let x is an element of L; then $x \in A$ and $x \in (B \cup C)$. x must be in A but can be a member of either B or C. \\
                    Let $x \in B$, then $x in (A \cap B)$ and thus $x \in (A \cap B) \cup (A \cap C)$. Similarly, let $x \in C$, then $x in (A \cap C)$ and thus, $x \in (A \cap B) \cup (A \cap C)$.
                    Therefore $L \subseteq R$.
                \item
                    Let $x \in R$. Then $x \in (A  \cap B)$ or $x \in (A \cap C)$. \\
                    If, $x \in (A \cap B)$, then $x \in A$ and $x \in B$. With this, $x \in A \cap (B \cup C)$. 
                    Similarly, it $x \in (A \cap C)$ then $x \in A$ and $x \in C$. This emplies $x \in A \cap (B \cup C)$.
                    Therefore $R \subseteq L$. \\
                    Hence we established $L = R$ \\
            \end{enumerate}
        \item
            $A \cap (A \cup B) = A$ \\
            Let, x in $A \cap (A \cup B)$, then because of intersection $x \in A$.\\
            Similarly, let $x \in A$ then, $x \in (A \cup C)$ and thus, $x \in A\cap (B \cup C)$.
        \item
            $A \cup (A \cap B) = A$ \\
            Let $x \in A \cup (A \cap B)$, with this we can say, $x \in A$. \\
            Similarly, let $x \in A$, then by $x \in A \cup (A \cap B) $.
            \pagebreak
        \item
            $A - (B \cap C) = (A - B)\cup (A - C)$\\
            Let, $L = A -(B \cap C)$ and $R = (A - B) \cup (A - C)$\\ 
            Let, $x \in L$, then $x \in A$ and $x \not\in (B \cap C)$. Thus x cannot be in B and C both.
            When $x \in B$ implies, $x \in (A - C)$ and thus $x \in R$. Similarly, when $x \in C$ implies $ x \in (A - B)$ and thus $x \in R$.
            When x is not in both B and C, then too $x \in R$. Thus $L \subseteq R$.\\
            Similarly, let $x \in R$, then $x \in A$, similarly $x \not\in (B \cap C)$. Thus, $x \in L$. And $R \subseteq L$.
            From this we have $L = R$.
    \end{enumerate}
\item
            Let $S = \{a, b, c, d\}.$
            \begin{enumerate}
                \item
                    What partitions of S has the fewest numbers? The most members? \\
                    The partition of S that has the fewest members is \{\{a, b, c, d\}\}, a set containing S. Similarly, the partition of S has the most members when it is a set of subsets of S such that each has one element of S, that is \{\{a\}, \{b\}, \{c\}, \{d\}\}.
                \item
                    List all the partitions of S with exactly two members. \\
                    They are: \\
                        \{\{a, b\}, \{c, d\}\}, \{\{a, c\}, \{b, d\}\}, \{\{a, d\}, \{b, c\}\}, \{\{a\}, \{b, c, d\}\}, \{\{b\}, \{a, c, d\}\}, \{\{c\}, \{a, b, d\}\}, \{\{d\}, \{a, b, c\}\}
            \end{enumerate}
\end{enumerate}

\end{document}
